\documentclass[10pt]{article}
\usepackage[a4paper, margin=1in]{geometry}
\usepackage{hyperref}
\usepackage{setspace}
\usepackage[style=iso]{datetime2}
\usepackage{amsmath}
\usepackage{eufrak}
\usepackage{amssymb}
\usepackage{mathtools}

\newcommand{\defeq}{\vcentcolon=}

\title{Automatic Differentiation Variational Inference - Mean Field Simplification}
\author{Harshad Deo \\ 
  \href{mailto:harshad@moreficent.com}{harshad@moreficent.com} \\ 
  \href{mailto:harshad@simianquant.com}{harshad@simianquant.com}
}
\date{}

\setlength{\parindent}{0cm}

\hypersetup{
  colorlinks,
  citecolor=blue,
  filecolor=blue,
  linkcolor=blue,
  urlcolor=blue
}
  
\setlength{\parskip}{\baselineskip}%
  
\begin{document}
  
\maketitle

This monograph is a follow up to the article introducing Auto Differentiation Variational Inference and develops simplifying
assumptions that allow the technique to be used to train large, complex models over large datasets, for example Bayesian 
Deep Learning.

\section*{Decoupling The Latent Space}

The full rank approximation, developed in the previous monograph, allows all the latent variables ($\theta$) to be coupled
with each other through the correlation matrix of the variational approximation. As a result, the number of parameters
required to model the latent space grows quadratically with the size of the latent space. This makes it intractable for 
use in all but the simplest problems, with a few thousand latent variables. 

The mean field approximation assumes that all the latent variables are independent. Therefore, the number of parameters
required to model the latent space grows linearly with the size of the latent space, making it practical to scale the size
of the model. Formally,

\begin{align*}
  q(\zeta, \phi) \defeq& \mathcal{N}\big(\zeta; \mu, \text{diag}(\exp(\omega) ^2)\big) \\
  =& \prod_{k=1}^K \mathcal{N}(\zeta_k; \mu_k, \exp(\omega_k)^2)
\end{align*}

With this parameterization, the space of the variational parameters is given by:

\begin{equation*}
  \Phi = \{\mu_1, \ldots, \mu_k, \omega_1, \ldots, \omega_k\} = \mathbb{R}^{2K}
\end{equation*}

And the elliptical standardization is given by:

\begin{equation*}
  \eta_k = S_{\phi}(\zeta) = \frac{\zeta_k - \mu_k}{\exp \omega_k}
\end{equation*}

The correlation, if required, now needs to be modelled in explicitly as a function of the latent variables. For example, 
the Lewandowski-Kurowicka-Joe transform \footnote{Lewandowski, D., Kurowicka, D., \& Joe, H. (2009). Generating random correlation matrices based on vines and extended onion method. Journal of multivariate analysis, 100(9), 1989-2001.} can be used to explicitly model a correlation matrix for a subset of the latent
variables. 

\section*{Simplifying Entropy}


\section*{Graph Simplification}

\end{document}
