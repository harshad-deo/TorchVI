\documentclass[10pt]{article}
\usepackage[a4paper, margin=1in]{geometry}
\usepackage{hyperref}
\usepackage{setspace}
\usepackage[style=iso]{datetime2}
\usepackage{amsmath}
\usepackage{eufrak}
\usepackage{amssymb}
\usepackage{mathtools}

\newcommand{\defeq}{\vcentcolon=}

\title{Introduction to Automatic Differentiation Variational Inference}
\author{Harshad Deo \\ 
  \href{mailto:harshad@moreficent.com}{harshad@moreficent.com} \\ 
  \href{mailto:harshad@simianquant.com}{harshad@simianquant.com}
}
\date{}

\setlength{\parindent}{0cm}

\hypersetup{
  colorlinks,
  citecolor=blue,
  filecolor=blue,
  linkcolor=blue,
  urlcolor=blue
}
  
\setlength{\parskip}{\baselineskip}%
  
  
\begin{document}
  
\maketitle

Bayesian methods allow us to capture uncertainty while drawing inferences and making predictions from data. Instead of 
point estimates, they provide the machinery to generate a probability distribution over the inference targets. Aside from 
guarding against over-zealous interpretations of the analysis, quantifying the uncertainty is useful when the utility of 
the inference target is a non-linear function of its value. For example, if two assets, say a stock or a house, have the
same expected return but the second has a much higher variance (i.e.\ is more risky), most people would invest in the 
first. 

Formally, given observations $x$ and latent variables $\theta$, Bayesian methods attempt to construct the posterior 
distribution of the latent variables given the data, i.e.\ $p(\theta|x)$. For most practical problems, it is intractable 
to calculate the true posterior, therefore in practice, Bayesian methods are a suite of algorithms that attempt to 
approximate the true posterior. There are two families of approaches that are used:

\begin{enumerate}
  \item Those that construct a stationary stochastic process whose outputs are samples from the joint posterior, like 
    Markov Chain Monte Carlo and Hamiltonian Monte Carlo. These do not scale to large datasets and are tricky at 
    best to parallelize.
  \item Those that choose the best approximation to the true posterior from a known family of distributions. This method 
    of inferring the true posterior is called variational inference (VI).
\end{enumerate}

This monograph presents Automatic Differentiation Variational Inference (ADVI), a new VI algorithm
developed by Kucukelbir at al \footnote{Kucukelbir, Alp, et al. ``Automatic differentiation variational inference." 
The Journal of Machine Learning Research 18.1 (2017): 430-474.} that overcomes the shortcomings of earlier VI approaches,
with a focus on using ADVI to fit complex models to large datasets. 

\section*{Theoretical Development}

Variational approaches haven't been frequently used in practice because each problem required the construction of a new
variational scheme and optimization procecure. ADVI aims to automate this process. 

\begin{enumerate}
  \item ADVI transforms any applicable probability model into one over unconstrained real latent variables. Since all the 
  latent variables are defined in the same space, ADVI can use a single variational family for all models. 
  \item ADVI expresses the gradient of the objective function as an expectation over a standard gaussian. This allows
  the gradient to be efficiently computed using Monte Carlo sampling. 
\end{enumerate}

ADVI is applicable to all differentiable probability models. Given a probability parameterized by a $K$ dimensional 
latent space $\theta$, and its support given by:

\begin{equation}
  supp(p(\theta)) = \{\theta | \theta \in \mathbb{R}^K \, \text{and} \, p(\theta) > 0 \} \subseteq \mathbb{R} ^ K \notag
\end{equation}

ADVI is applicable if $\theta$ is continuous and the gradient of the log joint density, $\nabla_{\theta}\log p(x, \theta)$, 
is defined within the support of the prior.

\subsection*{Objective Function}

Considering a family of approximating variational densities $q(\theta ; \phi)$, parameterized by $\phi \in \Phi$, 
variational inference finds the member of the family that minimizes the KL divergence to the posterior:

\begin{equation*}
  \phi^* = \underset{\phi \in \Phi}{\text{argmin}}\, KL(q(\theta ; \phi) || p(\theta | x)) \quad \text{such that} \quad supp(q(\theta ; \phi)) \subseteq supp(p(\theta | x))
\end{equation*}

The optimised $q(\theta; \phi^*)$ serves as an approximation to the posterior. 

\begin{align*}
  KL(q(\theta ; \phi) || p(\theta | x)) \defeq &\mathbb{E}_{q(\theta)}\big[\log q(\theta;\phi) - \log p(\theta|x)  \big] \\
  =& \mathbb{E}_{q(\theta)}\big[\log q(\theta;\phi) - \log p(\theta, x) + \log p(x)\big] \\
  =& \mathbb{E}_{q(\theta)}\big[\log q(\theta;\phi) - \log p(\theta, x) \big] + \log p(x)
\end{align*}

Since the KL divergence is itself dependent on the posterior density, it cannot be directly minimized. Instead, we try 
to maximise the Evidence Lower Bound (ELBO), given by:

\begin{equation*}
  \mathcal{L}(\phi) = \mathbb{E}_{q(\theta)}\big[\log p(x, \theta) - \log q(\theta;\phi) \big]
\end{equation*}

The negative of the ELBO is equal to the KL divergence, upto an additive constant. Therefore maximising the ELBO is 
equivalent to minimising the KL divergence. Therefore the target objective function is given by:

\begin{equation*}
  \phi^* = \underset{\phi \in \Phi}{\text{argmax}}\, \mathcal{L}(\phi) \quad \text{such that} \quad supp(q(\theta ; \phi)) \subseteq supp(p(\theta | x))
\end{equation*}

\subsection*{Unconstrained Objective Function}

The first step is to remove the domain constraint of the objective function. Define a bijective differentiable function $T$:

\begin{align*}
  &T : supp(p(\theta)) \to \mathbb{R}^K \\
  &\zeta = T(\theta)
\end{align*}

The transformed joint density now has the representation:

\begin{equation*}
  p(x, \zeta) = p\big(x, T^{-1}(\zeta)\big)|\det J_{T^{-1}}(\zeta)| 
\end{equation*}

To illustrate, consider the three most common constraint transforms as they are implemented in Stan:\footnote{Stan Development Team. 2020. Stan Modeling Language Users Guide and Reference Manual, 2.25. https://mc-stan.org}

\begin{itemize}
  \item $\theta > a \implies \zeta(\theta) = \log(\theta - a)$
  \item $\theta < b \implies \zeta(\theta) = \log(b - \theta)$
  \item $a < \theta < b \implies \zeta(\theta) = \log\big(\frac{\theta - a}{b - \theta}\big)$
\end{itemize}

With this transformation, the unconstrained objective function can be expressed as:

\begin{equation*}
  \phi^* = \underset{\phi \in \Phi}{\text{argmax}} \, \mathbb{E}_{q(\zeta)}\big[\log p(x, T^{-1}(\zeta)) + \log |\det J_{T^{-1}}(\zeta)| - \log q(\zeta, \phi) \big]
\end{equation*}

\end{document}
