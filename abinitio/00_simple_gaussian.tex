\documentclass[10pt]{article}
\usepackage[a4paper, margin=1in]{geometry}
\usepackage{hyperref}
\usepackage{setspace}
\usepackage{amsmath}
\usepackage{eufrak}
\usepackage{amssymb}
\usepackage{mathtools}

\newcommand{\defeq}{\vcentcolon=}

\title{Ab'Initio ADVI \\ Simple Gaussian}
\author{Harshad Deo \\ 
  \href{mailto:harshad@moreficent.com}{harshad@moreficent.com} \\ 
  \href{mailto:harshad@simianquant.com}{harshad@simianquant.com}
}
\date{}

\setlength{\parindent}{0cm}

\hypersetup{
  colorlinks,
  citecolor=blue,
  filecolor=blue,
  linkcolor=blue,
  urlcolor=blue
}
  
\setlength{\parskip}{\baselineskip}%
  
  
\begin{document}
  
\maketitle

Consider a Gaussian random variable with an unknown mean $\theta$ and known variance $\sigma_0$ with a flat prior on
the mean. Formally:

\begin{align*}
  X &\sim \mathcal{N}(\theta, \sigma_0^2)\\
  p(\theta) &\varpropto 1
\end{align*}

Since the domain of $\theta$ is not constrained, 

\begin{align*}
  \zeta &= T(\theta) = \theta \\
  \theta &= T^{-1}(\zeta) = \zeta
\end{align*}

Therefore the log determinant of the jacobian of the inverse transform evaluates to 0. Approximating $\zeta$ with a 
Gaussian distribution with mean $\mu$ and scale $\exp(\omega)$, the approximated variational space is given by:

\begin{equation*}
  \Phi = \{\mu, \omega\} = \mathbb{R}^2
\end{equation*}

Given the variational approximation, the elliptical standardization is given by:

\begin{align*}
  \eta &= S_{\phi}(\zeta) = \frac{\zeta - \mu}{\exp(\omega)} \\
  \zeta &= S_{\phi}^{-1}(\eta) = \mu + \eta \exp(\omega)
\end{align*}

Therefore the objective function is given by:

\begin{align*}
  \mu^*, \omega^* &= \underset{\mu, \omega}{\text{argmax}}\,\mathbb{E}_\eta\big[\log p(x, \mu + \eta\exp(\omega)) \big] + \omega \\
  &= \underset{\mu, \omega}{\text{argmax}}\,\mathbb{E}_\eta\big[\log p(x | \mu + \eta\exp(\omega)) + \log p(\mu + \eta \exp(\omega))] + \omega \\
  &= \underset{\mu, \omega}{\text{argmax}}\,\mathbb{E}_\eta\big[\log p(x | \mu + \eta\exp(\omega))] + \omega, \quad \text{since the prior on $\theta$ is flat}
\end{align*}

\end{document}
